\documentclass{beamer}
\usepackage{booktabs}
\usepackage{graphicx}
\usepackage{hyperref}
\usepackage{xcolor}
\usepackage{multirow}
\usepackage{listings}

\usetheme{Warsaw}
\usecolortheme{dolphin}

% Define colors
\definecolor{darkgreen}{RGB}{0,128,0}
\definecolor{darkred}{RGB}{128,0,0}

\setbeamertemplate{footline}{%
  \leavevmode%
  \hbox{%
    \begin{beamercolorbox}[wd=0.5\paperwidth,ht=2.5ex,dp=1ex,left]{author in head/foot}%
      \hspace{0.6cm}Battiston, Faccio, Magnabosco, Viler
    \end{beamercolorbox}%
    \begin{beamercolorbox}[wd=0.5\paperwidth,ht=2.5ex,dp=1ex,right]{date in head/foot}%
      \insertshorttitle\hspace*{2em}\insertframenumber{} / \inserttotalframenumber\hspace*{2ex}%
    \end{beamercolorbox}%
  }%
  \vskip0pt%
}

\title{\textbf{Hackathon Zucchetti}}
\author{Nicola Battiston, Christian Faccio, \\ Manuel Magnabosco, Christian Viler}
\date{April 6, 2025}

\begin{document}

\frame{\titlepage}

\begin{frame}{Approccio utilizzato}
    Some initial content.
    
    \pause  % creates the next overlay
    
    \uncover<2->{%
        \begin{block}{Outcoming Window}
            This block appears from the second overlay onward.
            You can put any content you like here.
        \end{block}
    }
\end{frame}

\begin{frame}{Ragionamento seguito}
    \textbf{1 - Previsione delle domande di mercato}
    \vspace{0.2cm}
    \begin{enumerate}
        \item Classificazione dei gruppi country-product in \textbf{dismissed, intervals, regular}
        \vspace{0.2cm}
        \item Pulizia dei dati \textbf{(outliers, normalizzazione)}
        \vspace{0.2cm}
        \item Applicazione del relativo modello di previsione:
        \begin{itemize}
            \item Dismissed $\to$ Dummy Regressor (predice sempre 0)
            \item Intervals $\to$ SARIMAX model 
            \item Regular $\to$ SARIMAX model
        \end{itemize}
        \vspace{0.2cm}
        \item Previsione della domanda per l'anno 2024
    \end{enumerate}
\end{frame}

\begin{frame}{Ragionamento seguito}
    \textbf{2 - Bilanciamento produzione tra stabilimenti}
    \begin{enumerate}
        \item Utilizzo della libreria \textbf{PuLP} per la programmazione lineare
        \item Sistemato i dati in dei dizionari per modificare i parametri del problema
        \item Inserita la funzione obiettivo e i vincoli
        \item Risoluzione del problema
    \end{enumerate}
\end{frame}

\begin{frame}{Ragionamento seguito}
    \textbf{3 - Ottimizzazione Costi di Produzione e Trasferimento}
    \begin{enumerate}
        \item Lettura dei dati di input
        \item Definizione dei costi di produzione e trasferimento
        \item Configurazione dei vincoli di capacità e domanda
        \item Risoluzione del modello utilizzando la libreria \textbf{PuLP} e analisi dei risultati
    \end{enumerate}
\end{frame}

\begin{frame}{Sfide e soluzioni}




\end{frame}


\begin{frame}{Sfide e soluzioni - Problema 2}
    

    \vspace{0.3cm}
    \begin{enumerate}
        \item \textbf{Saturazione per priorità:} abbiamo prodotto un prodotto alla volta, saturando la capacità di un singolo paese fino al limite, per poi passare al successivo.
        \vspace{0.2cm}
        \item \textbf{Distribuzione proporzionale:} abbiamo assegnato la produzione in modo bilanciato, facendo lavorare proporzionalmente tutte le risorse disponibili.
    \end{enumerate}

    \vspace{0.4cm}
    Entrambi gli approcci hanno però mostrato un margine d’errore elevato e risultati non ottimali.

    \vspace{0.4cm}
    Per questo ci siamo spostati verso la \textbf{programmazione lineare}.
\end{frame}

\begin{frame}{Sfide e soluzioni - Problema 3}
    L'approccio al \textbf{Problema 3} si fonda sul \textbf{Problema 2}, riutilizzando il modello, ottimizzando import / export come funzione obiettivo.

    \vspace{0.4cm}
    In particolare non abbiamo riscontrato particolari difficoltà, poiché la natura del problema è analoga a quella del precedente.

    \vspace{0.4cm}
    A differenza del secondo caso, non avevamo la certezza di aver raggiunto un \textbf{minimo globale}, quindi abbiamo esplorato soluzioni alternative.

    \vspace{0.4cm}
    Non trovando una soluzione migliore abbiamo ipotizzato di essere vicino alla \textbf{soluzione ottima} o in essa.
\end{frame}

\begin{frame}{Organizzazione del lavoro}
    Per decidere l'approccio al \textbf{Problema 1} abbiamo risolto con un brainstorming. 
    Trovata la strategia più efficace, i compiti sono stati suddivisi in base alle competenze di ognuno:

    \vspace{0.4cm}
    \begin{itemize}
        \item \textbf{Christian Faccio} ha lavorato sul \textbf{Problema 1}, cercando i modelli predittivi migliori per ridurre la \textit{Loss}.
        \item \textbf{Nicola} ha coordinato e guidato il sotto-gruppo verso la migliore metodologia dopo il confronto di idee per gli altri problemi.
        \item \textbf{Christian Viler} e \textbf{Manuel} Con \textbf{Nicola} hanno scritto l'algoritmo per la classificazione dei gruppi del \textbf{Problema 1}.
        In modo autonomo, hanno sviluppato sotto-funzioni per i programmi 2 e 3. Hanno gestito i tempi delle scadenze.
    \end{itemize}

    \vspace{0.4cm}
    La presentazione è stata pensata e strutturata da tutti i componenti del gruppo.
\end{frame}



\end{document}