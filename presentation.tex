\documentclass{beamer}
\usepackage{booktabs}
\usepackage{graphicx}
\usepackage{hyperref}
\usepackage{xcolor}
\usepackage{multirow}
\usepackage{listings}

\usetheme{Warsaw}
\usecolortheme{dolphin}

% Define colors
\definecolor{darkgreen}{RGB}{0,128,0}
\definecolor{darkred}{RGB}{128,0,0}

\setbeamertemplate{footline}{%
  \leavevmode%
  \hbox{%
    \begin{beamercolorbox}[wd=0.5\paperwidth,ht=2.5ex,dp=1ex,left]{author in head/foot}%
      \hspace{0.6cm}Battiston, Faccio, Magnabosco, Viler
    \end{beamercolorbox}%
    \begin{beamercolorbox}[wd=0.5\paperwidth,ht=2.5ex,dp=1ex,right]{date in head/foot}%
      \insertshorttitle\hspace*{2em}\insertframenumber{} / \inserttotalframenumber\hspace*{2ex}%
    \end{beamercolorbox}%
  }%
  \vskip0pt%
}

\title{\textbf{Hackathon Zucchetti}}
\author{Nicola Battiston, Christian Faccio, \\ Manuel Magnabosco, Christian Viler}
\date{April 6, 2025}

\begin{document}

\frame{\titlepage}

\begin{frame}{Approccio utilizzato}
    Some initial content.
    
    \pause  % creates the next overlay
    
    \uncover<2->{%
        \begin{block}{Outcoming Window}
            This block appears from the second overlay onward.
            You can put any content you like here.
        \end{block}
    }
\end{frame}

\begin{frame}{Ragionamento seguito}
    \textbf{1 - Previsione delle domande di mercato}
    \vspace{0.2cm}
    \begin{enumerate}
        \item Classificazione dei gruppi country-product in \textbf{dismissed, intervals, regular}
        \vspace{0.2cm}
        \item Pulizia dei dati \textbf{(outliers, normalizzazione)}
        \vspace{0.2cm}
        \item Applicazione del relativo modello di previsione:
        \begin{itemize}
            \item Dismissed $\to$ Dummy Regressor (predice sempre 0)
            \item Intervals $\to$ SARIMAX model 
            \item Regular $\to$ SARIMAX model
        \end{itemize}
        \vspace{0.2cm}
        \item Previsione della domanda per l'anno 2024
    \end{enumerate}
\end{frame}

\begin{frame}{Ragionamento seguito}
    \textbf{2 - Bilanciamento produzione tra stabilimenti}
    \begin{enumerate}
        \item Utilizzo della libreria \textbf{PuLP} per la programmazione lineare
        \item Sistemato i dati in dei dizionari per modificare i parametri del problema
        \item Inserita la funzione obiettivo e i vincoli
        \item Risoluzione del problema
    \end{enumerate}
\end{frame}

\begin{frame}{Ragionamento seguito}
    \textbf{3 - Ottimizzazione Costi di Produzione e Trasferimento}
    \begin{enumerate}
        \item Lettura dei dati di input
        \item Definizione dei costi di produzione e trasferimento
        \item Configurazione dei vincoli di capacità e domanda
        \item Risoluzione del modello utilizzando la libreria \textbf{PuLP} e analisi dei risultati
    \end{enumerate}
\end{frame}

\end{document}